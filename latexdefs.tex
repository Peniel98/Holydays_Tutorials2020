\newcommand{\Tr}{\mathrm{Tr}} % Trace of a matrix
\newcommand{\diag}{\mathrm{diag}} % Diagonal of a matrix

% Défintion des commandes raccourcies personnelles
\newcommand{\bls}[1]{\boldsymbol{#1}} % use boldsymbol
\newcommand{\mt}[1]{\mathtt{#1}} %  Use mathtt
\newcommand{\mc}[1]{\mathcal{#1}} % Use mathcal
\newcommand{\mb}[1]{\mathbb{#1}} % Use mathbb
\newcommand{\mr}[1]{\mathrm{#1}} % Use mathrm
%\newcommand{\gr}[1]{\boldsymbol{#1}}} % use boldsymbol
%
\newcommand{\ot}{\otimes} 
\newcommand{\sq}[1]{\sqrt{#1}}
\newcommand{\fr}[2]{\frac{#1}{#2}} 
\newcommand{\bra}[1]{\langle#1|} 
\newcommand{\ket}[1]{|#1\rangle} 
\newcommand{\av}[1]{\langle{#1}\rangle} %Average
\newcommand{\kb}[2]{|#1\rangle\langle#2|} %ketbra
\newcommand{\proj}[1]{\ket{#1}\bra{#1}} % Projecteur
\newcommand{\bk}[2]{\langle#1\ket{#2}} %braket
\newcommand{\mel}[3]{\bra{#1}#2\ket{#3}} %Matrix element
\newcommand{\ddt}[1]{\frac{\mathrm{d}#1}{\mathrm{d}t}}  %derivation par à t
\newcommand{\ddx}[1]{\frac{\mathrm{d}#1}{\mathrm{d}x}}  %derivation par à x
\newcommand{\dd}[2]{\frac{\mathrm{d}#1}{\mathrm{d}#2}}  %derivation par à 
\newcommand{\dx}{\mathrm{d}x}  %derivation par à x
\newcommand{\dpp}{\mathrm{d}p}  %derivation par à p
\newcommand{\dpt}[1]{\frac{\mathrm{\partial}#1}{\mathrm{\partial}t}} %derivation partielle par rapport à t
\newcommand{\dpx}[1]{\frac{\mathrm{\partial}#1}{\mathrm{\partial}x}} %derivation partielle par rapport à x
\newcommand{\dpz}[1]{\frac{\partial#1}{\partial z}} %derivation partielle par rapport à z
%
\newcommand{\tcb}[1]{\textcolor{blue}{#1}} % textcolor{blue}{#1}
\newcommand{\tcr}[1]{\textcolor{red}{#1}} % textcolor{red}{#1}
\newcommand{\tct}[1]{\textcolor{teal}{#1}} % textcolor{teal}{#1}
%
\newcommand{\hh}{\mathcal{H}} % Espace de Hilbert
\newcommand{\X}{\mathtt{X}} %Matrice de Pauli X
\newcommand{\Y}{\mathtt{Y}} %Matrice de Pauli Y
\newcommand{\Z}{\mathtt{Z}} %Matrice de Pauli Z
\newcommand{\W}{\mathtt{W}} %Operateur de Walsh-Hadamard W%
\newcommand{\U}{\mathtt{U}} %Operateur U quelconque%
\newcommand{\R}{\mathtt{R}} %Operateur R quelconque%
\newcommand{\PS}{\mathtt{P}} %Operateur P quelconque%
\newcommand{\HH}{\mathtt{H}} %Opérateur Hamiltonien H
\newcommand{\CX}{\mathtt{CX}} %Operateur CX %
\newcommand{\CZ}{\mathtt{CZ}} %Operateur CZ %
\newcommand{\CU}{\mathtt{CU}} %Operateur CU %
\newcommand{\CR}{\mathtt{CR}} %Operateur CR %
\newcommand{\BS}{\mathtt{BS}} %Operateur BS %
%
\newcommand{\T}{\mathtt{T}} %Operateur T quelconque%
\newcommand{\D}{\mathtt{D}} %Operateur D quelconque%
\newcommand{\G}{\mathtt{G}} %Operateur G quelconque%
\newcommand{\E}{\mathtt{E}} %Operateur E quelconque%
\newcommand{\K}{\mathtt{K}} %Operateur K quelconque%
\newcommand{\V}{\mathtt{V}} %Operateur K quelconque%
\newcommand{\N}{\mathtt{N}} %Operator N
\newcommand{\M}{\mathtt{M}} %Operator M
\newcommand{\A}{\mathtt{A}} %Operator A
\newcommand{\B}{\mathtt{B}} %Operator B
\newcommand{\Q}{\mathtt{Q}} %Operateur Q quelconque%
\newcommand{\Oo}{\mathtt{O}} %Operateur O quelconque%
\newcommand{\FF}{\mathtt{F}} %Operateur F quelconque%
\newcommand{\St}{\mathtt{S}} %Operateur S quelconque%
\newcommand{\I}{\mathbb{I}} %Opérateur Identité I
\newcommand{\OO}{\mathbb{O}} %Opérateur Nul O
\newcommand{\KK}{\mathbb{K}} %Superopérateur K%
\newcommand{\F}{\mathbb{F}} %Espace F
\newcommand{\C}{\mathbb{C}} %Espace C
%
\newcommand{\ad}{a^\dag} % Operateur bosonique adjoint
\newcommand{\nb}{\bar{n}} % n bar (thermal state)
\newcommand{\z}{\zeta} %Symbole \zeta
\newcommand{\hbd}{\frac{\hbar}{2}} %
\newcommand{\br}{\bls{r}} % r bold
\newcommand{\bkk}{\bls{k}} % k bold
\newcommand{\bA}{\bls{A}} % A bold
\newcommand{\bE}{\bls{E}} % E bold
\newcommand{\bB}{\bls{B}} % B bold
\newcommand{\bhvr}{\bls{\hat{\varepsilon}}}
%
%Encadrer les equations en couleurs
\newcommand{\tequation}[1]
{\begin{equation}\tcbhighmath{#1}
\end{equation}}

\newcommand{\talign}[1]
{\begin{empheq}[box=\tcbhighmath]{align}
#1
\end{empheq}}

%quantikz
\newcommand{\tquantikz}[1]
{\begin{quantikz}
#1
\end{quantikz}}
%
\newcommand{\ttquantikz}[1]
{\begin{tikzcd}[ampersand replacement=\&]
#1
\end{tikzcd}}

